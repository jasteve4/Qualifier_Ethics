\documentclass[12pt]{article}

\usepackage{blindtext}

\usepackage{fancyhdr}
\usepackage{lastpage}

\usepackage[paper=letterpaper,
            %includefoot, % Uncomment to put page number above margin
            marginparwidth=0in,     % Length of section titles
            marginparsep=0in,       % Space between titles and text
            margin=1in,               % 1 inch margins
            includemp]{geometry}

\usepackage{natbib}
\usepackage{hyperref}
\setlength{\bibsep}{0.0pt} %reference spacing

\usepackage{titlesec}
\titlespacing\section{2pt}{12pt plus 4pt minus 2pt}{2pt plus 2pt minus 2pt}
%\titlespacing{command}{left spacing}{before spacing}{after spacing}[right]
\begin{document}

\begin{center}
{\Large Qualifying Exam} \\[.1in]
{\large Research Ethics, Writing Skills and Presentation Skills} \\[.1in]
{\large Joshua Stevens}
\end{center}


\section{Introduction}

Research is the process of adding knowledge to the global community. The last step a researcher takes on a topic is to publish and present one's work.  For publishing a body of work, a researcher has to write a comprehensive paper on what the body of work adds to the research community and not to undermine the collaborative community effort. For presenting one's work, the researcher needs to be able to convey details of the research to the community orally. The following section will cover research ethics, writing skills, and presentation skills need to accomplish this goal.

\section{Research Ethics}

What is research ethics, and how does one write ethically? To answer the first question, W. Booth states the definition of ethics: "the forging of bonds that create a community and the moral choices we face when we act in that community.\cite{Booth}"  So writing ethically in a research paper means to add to the overall conversion. This concept implies that the work presented in a paper should be building upon research already done or making sure that the new idea is, in fact, a new idea. So, do not plagiarise\cite{Michaelson}. According to Herbert Michaelson, plagiarism can manifest in two ways: deliberate or unintentional\cite{Michaelson}. Both are equally bad.  The intentional form and the unintentional form will result in misleading the reader. Misleading the reader can occur in the following methods: claiming other researcher ideas as the writer's own, leaving out details that might discredit or devalue the writer's work\cite{Michaelson}.        



\section{Technical Writing Skills}

The main focuses of technical writing hinges on the three crucial sections: abstract, introduction, and conclusion. The abstract of a paper is used to convey the main ideas of a given topic. According to Herbert Michaelson, an abstract is laid out in one of the following ways: indicative, informative, and informative-indicative\cite{RefWorks:Michaelson}. For the indicative style abstract, it is used to preview the contents covered in the paper\cite{RefWorks:Michaelson}. For the informative style, it is used to discuss the main findings of the paper's work briefly\cite{RefWorks:Michaelson}. For the informative-indicative style, it is a combination of the two previous styles. It provides the results, conclusions, and an overview of the paper\cite{RefWorks:Michaelson}. 

Unlike the abstract, the introduction gives an in-depth look at the material. Herbert Michaelson states that an introduction needs to have the following content: provide the purpose, state the problem, give background on the material, and list all contributors\cite{RefWorks:Michaelson}. It needs to give a scope, provide the rationale, and provide an order of the content to follow\cite{RefWorks:Michaelson}. In addition to having this content, W. Booth states that an introduction needs to follow a specified pattern: setting the context, indicating the problem, and the resolution to that issue\cite{Booth}. With this pattern and content, the introduction will be able to give the reader a good ideal of the material and scope of the topic.   

The conclusion is similar to the introduction but differs by putting more emphasis on the resolution of the topic. W. Booth states that a conclusion serves two purposes, to review the papers' main points and to elaborate on ideas that can lead to further research\cite{Booth}. In the review section, Herbert Michaelson says that it needs to contain the following content: a restatement of the problem, the resolution to the issue, and primal results of that resolution\cite{RefWorks:Michaelson}. After reading the conclusion, the reader will have a good sense of the issue and how it can be solved or how to solve it in future research. 


\section{Presentation Skills}

What makes a presentation memorable and what elements are necessary to produce such a presentation. It is memorable if and only if the audience can retain and understand the content. So, knowledge of the audience is essential. Different audiences require different scopes, so the scope needs to be at a digestible level and needs to be conveyed at the beginning of the presentation. R. Anholt says conveying the scope at the beginning will engage the audience form the start\cite{Anholt}. If mishandled, then the audience will be lost and unable to follow from the start. All presentation have a begin, middle, and end\cite{Anholt}\cite{Alley}. The beginning lays the groundwork, the middle conveys the main points of the topic, and the end reinforces the main points\cite{Anholt}\cite{Alley}. When in the middle section of the presentation, it is necessary to change the scope. R. Anholt calls this zooming in and zooming out\cite{Anholt}. When transitioning from one sub-topic to another, the presenter needs to zoom in at the beginning of the topic and then zoom out at the end\cite{Anholt}. Otherwise, the audience can become confused and lost during the transition. Knowing how far to zoom in is referred to as depth\cite{Alley}. The depth is one of the most changeling parts of creating a presentation\cite{Alley}. In M. Alley's book, he states that "depth is interwoven with the scope, which consists of the boundaries of the presentations.\cite{Alley}" Getting the depth right will allow the audience to stay engaged.


\section{Conclusions}

This paper has detailed how to publish and present one's work. The first section covers the ethics of research and how to make sure that one's research is adding to the global community. The second section covers skills need to publish one's work. The final section covers the skills need to present one's work.


\bibliographystyle{unsrt}
\bibliography{references}
\end{document}
